\documentclass[12pt]{scrartcl}

\setlength{\parindent}{0pt}
\setlength{\parskip}{.25cm}

\usepackage{graphicx}

\usepackage{xcolor}

\definecolor{darkred}{rgb}{0.5,0,0}
\definecolor{darkgreen}{rgb}{0,0.5,0}
\usepackage{hyperref}
\hypersetup{
  letterpaper,
  colorlinks,
  linkcolor=red,
  citecolor=darkgreen,
  menucolor=darkred,
  urlcolor=blue,
  pdfpagemode=none,
  pdftitle={CSCE 156 Lab Handout},
  pdfauthor={Christopher M. Bourke},
  pdfsubject={},
  pdfkeywords={}
}

\definecolor{MyDarkBlue}{rgb}{0,0.08,0.45}
\definecolor{MyDarkRed}{rgb}{0.45,0.08,0}
\definecolor{MyDarkGreen}{rgb}{0.08,0.45,0.08}

\definecolor{mintedBackground}{rgb}{0.95,0.95,0.95}
\definecolor{mintedInlineBackground}{rgb}{.90,.90,1}

%\usepackage{newfloat}
\usepackage[newfloat=true]{minted}
\setminted{mathescape,
               linenos,
               autogobble,
               frame=none,
               framesep=2mm,
               framerule=0.4pt,
               %label=foo,
               xleftmargin=2em,
               xrightmargin=0em,
               startinline=true,  %PHP only, allow it to omit the PHP Tags *** with this option, variables using dollar sign in comments are treated as latex math
               numbersep=10pt, %gap between line numbers and start of line
               style=default, %syntax highlighting style, default is "default"
               			    %gallery: http://help.farbox.com/pygments.html
			    	    %list available: pygmentize -L styles
               bgcolor=mintedBackground} %prevents breaking across pages
               
\setmintedinline{bgcolor={mintedBackground}}
\setminted[text]{bgcolor={mintedBackground},linenos=false,autogobble,xleftmargin=1em}
%\setminted[php]{bgcolor=mintedBackgroundPHP} %startinline=True}
\SetupFloatingEnvironment{listing}{name=Code Sample}
\SetupFloatingEnvironment{listing}{listname=List of Code Samples}


\title{CSCE 156 -- Computer Science II}
\subtitle{Lab 9.0 - JDBC in a Webapp I}
\author{~}
\date{~}

\begin{document}

\maketitle

\section*{Prior to Lab}

\begin{enumerate}
  \item Review this laboratory handout prior to lab.
  \item Make sure that the Albums database is installed and available 
  	in your MySQL instance on CSE
  \item Review the SQL and JDBC lecture notes
  \item Review a JDBC tutorial from Oracle: \\
	\url{http://download.oracle.com/javase/tutorial/jdbc/}
\end{enumerate}

\section*{Lab Objectives \& Topics}
Following the lab, you should be able to:
\begin{itemize}
  \item Write SQL queries for use in JDBC
  \item Make a JDBC connection, query and process a result set from a database
  \item Have some exposure to a multi-tiered application and a web 
  	application server
\end{itemize}

\section*{Peer Programming Pair-Up}

To encourage collaboration and a team environment, labs will be
structured in a \emph{pair programming} setup.  At the start of
each lab, you will be randomly paired up with another student 
(conflicts such as absences will be dealt with by the lab instructor).
One of you will be designated the \emph{driver} and the other
the \emph{navigator}.  

The navigator will be responsible for reading the instructions and
telling the driver what to do next.  The driver will be in charge of the
keyboard and workstation.  Both driver and navigator are responsible
for suggesting fixes and solutions together.  Neither the navigator
nor the driver is ``in charge.''  Beyond your immediate pairing, you
are encouraged to help and interact and with other pairs in the lab.

Each week you should alternate: if you were a driver last week, 
be a navigator next, etc.  Resolve any issues (you were both drivers
last week) within your pair.  Ask the lab instructor to resolve issues
only when you cannot come to a consensus.  

Because of the peer programming setup of labs, it is absolutely 
essential that you complete any pre-lab activities and familiarize
yourself with the handouts prior to coming to lab.  Failure to do
so will negatively impact your ability to collaborate and work with 
others which may mean that you will not be able to complete the
lab.  

\section*{JDBC in a Web Application}

In this lab you will familiarize yourself with the Java Database 
Connectivity API (JDBC) by finishing a simple, nearly complete 
retrieve-and-display web application and deploying it to an 
application server (Glassfish).  The design of the webapp is 
simple: it consists of an index page that loads album data 
via Ajax (Asynchronous JavaScript and XML) and displays it in
a table.  

It is not necessary to understand the details of the application 
(the HTML, JavaScript, Servlets, or application server).  The 
main goal of this lab is to give you some familiarity with JDBC 
and exposure to a multi-tiered application and web application 
server environment.

\section*{Getting Started}

\textbf{Note:} For this lab, you will need to use the JEE (Java 
Enterprise Edition) version of Eclipse, not the JSE (Java Standard 
Edition).  If you have this installed you should be able to use your
own machine, but if not you will need to use one of the lab computers.
In addition, it would be a good idea to reset your Albums database 
by rerunning the SQL script from a prior lab.

For this lab, you will need to use the JEE (Java Enterprise Edition) 
version of Eclipse, not the JSE (Java Standard Edition).  In Windows, 
click the start menu and enter ``Eclipse'', the ``Java EE Eclipse'' 
should show up, select this version.  You may use the same workspace 
as with the JSE (Java Standard Edition) version of Eclipse.

Clone the project code for this lab from GitHub using the URL, 
\url{https://github.com/cbourke/CSCE156-Lab09}.  Refer to Lab 
1.0 for instructions on how to clone a project from GitHub.	

\section*{Activities}

\subsection*{Modifying Your Application}

\begin{enumerate}
  \item You will first need to make changes to the 
  	\mintinline{java}{unl.cse.albums.DatabaseInfo} source file.  In 
	particular, change the login and password information to your
	MySQL credentials.  You can reset these by going to \url{http://cse.unl.edu/check}.
  \item The HTML, JavaScript, etc. has been provided for you.  Feel
    free to make modifications these files, but you should know what
    you are doing as changes can break functionality in other parts
    of the application.
  \item The application will not display any album data until you have
	completed the methods in the \mintinline{java}{Album} class.
	\begin{itemize}
	  \item \mintinline{java}{public static List<Album> getAlbumSummaries()} -- 
	  This method will query the database and get a complete list of all 
	  albums in the database.  It will create and populate 
	  \mintinline{java}{Album} objects and put them in a list 
	  which will then be returned.  This method will be used to 
	  generate the album table, so it doesn't need all information, just
	  a subset (see the documentation as to what is required).  
	  You should optimize your queries to only select the relevant columns.

	  \item \mintinline{java}{public static Album getDetailedAlbum(int albumId)} --
	  this method will query the database for the specific album with 
	  the given primary key and return an \mintinline{java}{Album}
	  instance with \emph{all} relevant data (band and its members, 
	  songs, etc.) specified. 

	  \item Important: do not forget to close your database resources 
      (especially connections) after you are finished using them.  
 	  \item A \mintinline{java}{Test} class has been provided for you 
      to test your \mintinline{java}{getAlbumSummaries()} method
      which you can also adapt to test your 
      \mintinline{java}{getDetailedAlbum()} method.  You should use 
      it to debug your methods before deploying your application.
    \end{itemize}
\end{enumerate}
    
\subsection*{Running Your Application}

When working with a Java Web Application you generally build
a Web Archive (WAR) file and \emph{deploy} it to an application server.  
For this lab, we'll ``deploy'' locally using Eclipse.  To start
your application follow these instructions.

\begin{enumerate}
  \item Change to the ``Java EE'' perspective in Eclipse
  \item Right click your project and select ``Run As'' 
  $\rightarrow$ ``Run on Server''
  \item Select ``J2EE Preview as localhost''
  \item After a brief startup, a preview browser window 
  should appear and you can interact with your application.
  \item To stop the server, hit the stop button in the terminal
  \item You may need to restart the server if you make any
  changes to your code.
\end{enumerate}

\subsection*{Completing Your Lab}

Complete the worksheet and have your lab instructor sign off on it.

\section*{Advanced Activities (Optional)}

\begin{enumerate}
  \item The album listing page utilizes a web framework called Bootstrap
  	(see \url{http://getbootstrap.com/}).  However, the album detail and 
	band detail pages do not.  Take this opportunity to learn about 
	Bootstrap and use it to add stylistic elements to these pages.
  \item Many JavaScript plugins are available to add additional 
    functionality to a plain HTML table (the ability to sort, pagination,
    column rearrangement, searching, filtering, etc.).  One of the best 
    plugins is datatables, a jQuery plugin available at 
    \url{http://datatables.net/}.  Download and incorporate datatable's 
    code into your project and add the appropriate JavaScript code 
    to make your Album table more dynamic.
\end{enumerate}

\end{document}
