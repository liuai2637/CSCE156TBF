\documentclass[12pt]{scrartcl}


\setlength{\parindent}{0pt}
\setlength{\parskip}{.25cm}

\usepackage{graphicx}

\usepackage{xcolor}

\definecolor{darkred}{rgb}{0.5,0,0}
\definecolor{darkgreen}{rgb}{0,0.5,0}
\usepackage{hyperref}
\hypersetup{
  letterpaper,
  colorlinks,
  linkcolor=red,
  citecolor=darkgreen,
  menucolor=darkred,
  urlcolor=blue,
  pdfpagemode=none,
  pdftitle={CSCE 156 Lab Handout},
  pdfauthor={Christopher M. Bourke},
  pdfsubject={},
  pdfkeywords={}
}

\definecolor{MyDarkBlue}{rgb}{0,0.08,0.45}
\definecolor{MyDarkRed}{rgb}{0.45,0.08,0}
\definecolor{MyDarkGreen}{rgb}{0.08,0.45,0.08}

\definecolor{mintedBackground}{rgb}{0.95,0.95,0.95}
\definecolor{mintedInlineBackground}{rgb}{.90,.90,1}

%\usepackage{newfloat}
\usepackage[newfloat=true]{minted}
\setminted{mathescape,
               linenos,
               autogobble,
               frame=none,
               framesep=2mm,
               framerule=0.4pt,
               %label=foo,
               xleftmargin=2em,
               xrightmargin=0em,
               startinline=true,  %PHP only, allow it to omit the PHP Tags *** with this option, variables using dollar sign in comments are treated as latex math
               numbersep=10pt, %gap between line numbers and start of line
               style=default, %syntax highlighting style, default is "default"
               			    %gallery: http://help.farbox.com/pygments.html
			    	    %list available: pygmentize -L styles
               bgcolor=mintedBackground} %prevents breaking across pages
               
\setmintedinline{bgcolor={mintedBackground}}
\setminted[text]{bgcolor={mintedBackground},linenos=false,autogobble,xleftmargin=1em}
%\setminted[php]{bgcolor=mintedBackgroundPHP} %startinline=True}
\SetupFloatingEnvironment{listing}{name=Code Sample}
\SetupFloatingEnvironment{listing}{listname=List of Code Samples}


\usepackage{tikz}
\usetikzlibrary{calc,shapes.multipart,chains,arrows}

\title{CSCE 156 -- Computer Science II}
\subtitle{Lab 14.0 - Stacks \& Queues}
\author{~}
\date{~}

\begin{document}

\maketitle

\section*{Prior to Lab}

\begin{enumerate}
  \item Review this laboratory handout prior to lab.
\end{enumerate}

\section*{Lab Objectives \& Topics}
Following the lab, you should be able to:
\begin{itemize}
  \item Understand how stacks and queues operate and how to use them
  \item Know how to implement stacks and queues using a linked list 
    data structure
  \item Know how to use stack and queue data structures in an application
  \item Optionally, you will have exposure to advanced queue usage 
    in a multi-threaded programming environment
\end{itemize}


\section*{Peer Programming Pair-Up}

To encourage collaboration and a team environment, labs will be
structured in a \emph{pair programming} setup.  At the start of
each lab, you will be randomly paired up with another student 
(conflicts such as absences will be dealt with by the lab instructor).
One of you will be designated the \emph{driver} and the other
the \emph{navigator}.  

The navigator will be responsible for reading the instructions and
telling the driver what to do next.  The driver will be in charge of the
keyboard and workstation.  Both driver and navigator are responsible
for suggesting fixes and solutions together.  Neither the navigator
nor the driver is ``in charge.''  Beyond your immediate pairing, you
are encouraged to help and interact and with other pairs in the lab.

Each week you should alternate: if you were a driver last week, 
be a navigator next, etc.  Resolve any issues (you were both drivers
last week) within your pair.  Ask the lab instructor to resolve issues
only when you cannot come to a consensus.  

Because of the peer programming setup of labs, it is absolutely 
essential that you complete any pre-lab activities and familiarize
yourself with the handouts prior to coming to lab.  Failure to do
so will negatively impact your ability to collaborate and work with 
others which may mean that you will not be able to complete the
lab.  

Clone the starter code for this lab from GitHub using the following
url: \url{https://github.com/cbourke/CSCE156-Lab14}.

\section*{Stacks}

Postfix notation (also known as Reverse Polish Notation) is a 
parenthesis-free way of writing arithmetic expressions where one 
places the operator symbol after its two operands.  For example, 
the expression \mintinline{text}{(3 + 2)} would be written 
\mintinline{text}{3 2 +}.  The result of this value multiplied by 
4 could be written \mintinline{text}{3 2 + 4 *}.  As another example, 
the expression \mintinline{text}{((3 + 2) * 4) / (5 - 1)} would be
written as: \mintinline{text}{3 2 + 4 * 5 1 - /}.  Manually computing
this would done as follows

\begin{minted}{text}
    3 2 + 4 * 5 1 - /
=>  5 4 * 5 1 - /
=>  20 5 1 - /
=>  20 4 /
=>  5
\end{minted}

The advantage of postfix notation is that no precedence rules are 
necessary as the order of operation is completely unambiguous.  
Many calculators, mathematical tools, and other systems utilize 
this notation.  Evaluation of postfix expressions can be done using 
a stack data structure.

Recall that a stack is a LIFO (Last-In First-Out) data structure.  
Stacks, like many data structures, can be implemented in many 
different ways.  In this lab, our implementation will utilize a 
linked list data structure.  This data structure has been completely 
implemented for you.  Most of the code needed to read-in a postfix 
expression and evaluate it has also been provided to you.  You will 
complete the application by implementing the stack data structure 
and use it in the postfix evaluation program.

\subsection*{Instructions}

\begin{enumerate}
  \item Implement the four methods in the 
    \mintinline{java}{unl.cse.stack.Stack} class (which uses the fully 
    implemented \mintinline{java}{LinkedList} class).
  \item You should test your implementation by creating a small 
    \mintinline{java}{main()} method and push/pop elements off your 
    stack to see if you get the expected behavior.
  \item Once your stack is implemented, modify the 
    \mintinline{java}{evaluateExpression()} method in the 
    \mintinline{java}{PostfixEvaluator} class as described in the 
    source code.  Run the program, complete your worksheet and have 
    a lab instructor check your work.
\end{enumerate}

\section*{Queues}

Queue data structures, in contrast to Stacks, are a FIFO (First-In 
First-Out) data structure.  Like stacks, they can easily be 
implemented using a linked list.  Queues have numerous applications; 
one such application is as a data buffer.  In many applications, 
processing a stream of data is expensive.  The data stream may be 
faster than an application can process it.  For this reason, incoming 
data is temporarily stored in a buffer (a queue).  The application 
then reads from the buffer and processes the data independent of the 
data stream. 

In this activity, you will implement and use a queue that acts as a 
data buffer for a plain text file.  The application reads an entire 
text file and stores lines into your queue for later processing.  The 
processing in this case is a human user that reads the text file 
page-by-page (by pressing enter).   In addition, the text file is 
not well-formatted for human readers as it contains very long lines.  
The application processes these lines and displays them page-by-page 
with a predefined limit on the number of characters per line so that it is 
more human-user friendly.

It will be your responsibility to complete the implementation of the 
\mintinline{java}{Queue} class and to use it properly in the 
\mintinline{java}{FileReader} class.

\subsection*{Instructions}

\begin{enumerate}
  \item Implement the four methods in the 
    \mintinline{java}{unl.cse.queue.Queue} class (which uses the fully 
    implemented \mintinline{java}{LinkedList} class)
  \item As before, you should test your implementation by creating a 
    small \mintinline{java}{main()} method and enqueue/dequeue elements 
    from your to see if you get the expected behavior.
  \item Once your queue is implemented, modify the 
    \mintinline{java}{FileReader} class as specified in the source 
    code (the \mintinline{java}{TODO} tasks) to use your queue to 
    process the text file.
\end{enumerate}    
    
\section*{JSON Validation}

Recall that JSON (JavaScript Object Notation) is a data exchange 
format that uses key-value pairs and opening/closing curly brackets 
to indicate sub-objects.  A small example:

\begin{minted}{json}
{
  "employees": [
    { 
      "firstName":"John", 
      "lastName":"Doe" 
    }, 
    { 
      "firstName":"Anna", 
      "lastName":"Smith"
    }, 
    { 
      "firstName":"Peter", 
      "lastName":"Jones" 
    }
  ],
  "deptIds": [1, 5, 4, 21],
  "department": "sales",
  "budget": 120000.00
}
\end{minted}

JSON validation involves determining if a particular string represents 
a valid JSON formatted string.  While not too complicated, full 
validation does require several complex rules that need to be checked.  
For this exercise, you will design and implement a JSON validator that 
will simply check that the opening and closing curly brackets, 
\mintinline{text}{{ }} and square brackets
\mintinline{text}{[ ]} are well-balanced.  That is, 
for every opening bracket there is a corresponding closing bracket 
of the same type.  Moreover, brackets of different types must not 
overlap.  For example, \mintinline{text}{ { [ } ] } would be invalid. 

\subsection*{Instructions}

\begin{enumerate}
  \item Open the \mintinline{java}{unl.cse.stack.JsonValidator} class.
    Some basic code has been implemented for you that grabs the 
    contents of a specified file
  \item Using an appropriate data structure, implement the 
    \mintinline{java}{validate()} method (you may change its signature 
    if you wish)
  \item Test your program on the 5 JSON data files in the 
    \mintinline{java}{data/} directory and answer the questions 
    on your worksheet
  \item \textbf{Extra}: time permitting; modify your program to also 
    check that double-quote characters (used to denote strings) are 
    well balanced.  Note that inside a string square and curly brackets 
    need \emph{not} be balanced (and should be ignored).  Moreover, 
    double quotes may appear inside of strings as long as they are 
    escaped: \mintinline{text}{\"}
\end{enumerate}

\section*{Submission}

We have included a test suite of unit tests written in JUnit 
(\url{https://junit.org/junit5/}) a popular unit testing framework for
Java.  Even though the test driver (in the \mintinline{text}{src/test}
source folder) has no main method, you can still run it in Eclipse and
get a report on how many of the tests passed, failed or resulted in 
an unexpected exception.  Be sure all of the unit tests pass before
submitting your source files through webhandin.  You can rerun this
test suite in the webgrader to ensure everything works.  


\section*{Advanced Activity (optional)}

Queues are often used in a common software pattern called a 
\emph{Consumer-Producer} pattern.  The pattern follows the idea that 
many independent producer agents (threads, processes, users, etc.) 
could be generating data or requests that must be processed or handled 
by independent consumer agents.  This pattern is very useful in 
improving system performance because it facilitates asynchronous 
behavior--producer and consumer agents can act independently of each 
other in a multi-threaded environment while requests are stored in a 
thread-safe queue.  Producers enqueue requests that do not require 
immediate processing and are free to continue in their execution.  
Consumers dequeue and process requests as they are able to, enabling 
asynchronous and independent processing.

Consider the following scenario.  A web browser posts a form request 
to a web application which processes the data, writes to a database, 
updates a log, and sends an email notification to the user and serves 
a response page back to the user.  If this process were synchronous, 
each action would have to wait for the previous action to terminate 
before it could begin even if each of the actions were not dependent 
on each other.  The end user would have to wait for all of the actions 
to terminate, giving a less than ideal user experience.  In a multi-user 
system, this wait time would only compound.

The advanced activity for this lab will be to familiarize you with the c
onsumer-producer pattern using Java.  We have mocked up a simple 
\mintinline{java}{Producer} class which makes an HTTPS connection to 
an API provided by Twitter (\url{http://www.twitter.com}) which serves 
a continuous stream of a subsample of twitter posts in a JSON 
(JavaScript Object Notation).  The \mintinline{java}{Producer} streams 
these tweets (represented as a JSON string) and places them in a 
\mintinline{java}{BlockingQueue<T>}, a thread-safe queue implementation.  
When threads dequeue elements from the queue if none are available, the 
queue ``blocks'' the thread (places it in a sleep status) until an 
element is available.

We have also designed a \mintinline{java}{Consumer} class that grabs 
elements off the queue and process them by parsing the JSON String 
and creating an instance of a Tweet class.  Subsequently, it outputs 
this tweet to the standard output.

The \mintinline{java}{PCDemo} creates one \mintinline{java}{Producer} 
instance and several Consumer instances (with ?names? for demonstration 
purposes) and starts each one in their own thread.  This is accomplished 
by making the \mintinline{java}{Producer} and \mintinline{java}{Consumer} 
classes implement the \mintinline{java}{Runnable} interface (there are 
alternatives).

Unfortunately, Twitter decided to change its API and require an 
API key to connect to its firehose.  If you modify and update the
code to work with their API (possibly using a library), inform
the instructor and share your code!

\end{document}
